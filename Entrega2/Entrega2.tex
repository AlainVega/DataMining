\documentclass[12pt, letterpaper]{article}

\usepackage{hyperref} % para manejar hipervinculos
\usepackage[spanish]{babel} % Para palabras en espanhol
\usepackage{graphicx} % LaTeX package to import graphics
\graphicspath{{../images/}} % configuring the graphicx package
\usepackage[spanish]{babel} % para que LaTeX corte palabras
\usepackage[dvipsnames]{xcolor} % color en el texto
\usepackage[backend=biber,style=numeric]{biblatex} % para la bibliografia.
\addbibresource{bibliografia.bib} % incluir el .bib
\hypersetup{ % configuracion de los hiper vinculos
    colorlinks=true,
    linkcolor=black,
    filecolor=black,      
    urlcolor=blue,
    citecolor=black, 
    pdftitle={Redes neuronales},
    pdfpagemode=FullScreen,
}
\usepackage{csquotes} % para que los textos citados esten con tipografia de acuerdo con las reglas de csquotes
\setcounter{biburlnumpenalty}{9000} % Para el line breaker en URL con numeros (que tambien afecta a la bibliografia)
\setcounter{biburllcpenalty}{9000} % Para el line breaker en URL con letras minuculas (que tambien afecta a la bibliografia)
\setcounter{biburlucpenalty}{9000} % Para el line breaker en URL con letras mayusculas (que tambien afecta a la bibliografia)

\begin{document}
% Portada
\begin{titlepage}
  \begin{center}
      \Large{Universidad Catolica "Nuestra señora de la Asuncion" \\
      Facultad de ciencias y tecnologia \\
      Complementos de Informatica}
      \includegraphics[width=0.8\textwidth]{UcaLogo.jpg}
      \LARGE{\textbf{Análisis exploratorio y selección de los datos 
      a utilizar para los modelos.}} \\
      \Large{Entrega 2 - Data mining}
      \vspace{1cm}
  \end{center}
      \large
      \textbf{Alumno: }Alain Vega \\
      \textbf{Profesor: }Wilfrido Felix Inchaustti Martínez
      \vfill
      \hfill{16 de Octubre del 2023}
\end{titlepage}


\newpage
\tableofcontents % indice
\newpage

% Contenidos
\section{Datos a utilizar}
Los datos a utilizar en el proyecto sera un archivo .csv 
(\textit{Comma Separated Values}) llamado \textit{heart.csv} 
que se obtuvo de la plataforma web \textit{Kaggle} \cite{dataset}
el cual de inicio contiene 12 columnas, las cuales:
\begin{enumerate}
    \item{\textbf{\textit{Age}}. edad del paciente} 
    \item{\textbf{\textit{Sex}}. sexo del paciente, donde:
    \begin{itemize}
        \item{\textbf{M}}: Masculino
        \item{\textbf{F}}: Femenino
    \end{itemize}
    }
    \item{\textbf{\textit{ChestPain}}. tipo de dolor en el pecho el cual puede ser: 
    \begin{itemize}
        \item{\textbf{TA}}: \textit{Typical Angina}
        \item{\textbf{ATA}}: \textit{Atypical Angina}
        \item{\textbf{NAP}}: \textit{Non-Anginal Pain}
        \item{\textbf{ASY}}: \textit{Asymptomatic}
    \end{itemize}
    }
    \item{\textbf{\textit{RestingBP}}. presión arterial en reposo 
    medido en mililitros de mercurio [mmHg]}
    \item{\textbf{\textit{Cholesterol}}. Colesterol serico 
    medido en miligramos por decilitro de sangre [mm/dl]}
    \item{\textbf{\textit{FastingBS}}. azúcar en sangre en ayunas
    medido en miligramos por decilitro [mg/dl]}
    \item{\textbf{\textit{RestingECG}}. resultados del electrocardiograma 
    en reposo, puede ser:
    \begin{itemize}
        \item{\textbf{Normal}}: normal
        \item{\textbf{ST}}: tener anomalía de la onda ST-T 
        (inversiones de la onda T y/o elevación o depresión del ST \(>\) 0,05 mV)
        \item{\textbf{LVH}}: muestra probable o definitiva hipertrofia ventricular 
        izquierda según los criterios de Estes
    \end{itemize}
    } 
    \item{\textbf{\textit{MaxHR}}. frecuencia cardíaca máxima alcanzada
    medido en pulsaciones por minuto [ppm]}
    \item{\textbf{\textit{ExerciseAngina}}. angina inducida por el ejercicio, puede ser:
    \begin{itemize}
        \item{\textbf{Y}}: Si
        \item{\textbf{N}}: No
    \end{itemize}
    }
    \item{\textbf{\textit{OldPeak}}. pico antiguo = ST}
    \item{\textbf{\textit{ST\_Slope}}. pendiente del segmento ST 
    durante un ejercicio físico máximo en una prueba de 
    esfuerzo cardíaco. Puede ser:
    \begin{itemize}
        \item{\textbf{Up}}: ascendente
        \item{\textbf{Flat}}: plano
        \item{\textbf{Down}}: descendente
    \end{itemize}
    }
    \item{\textbf{\textit{HeartDisease}}. sufrio de insuficiencia cardiaca
    \begin{itemize}
        \item{\textbf{1}}: insuficiencia cardiaca
        \item{\textbf{0}}: normal
    \end{itemize}
    }
\end{enumerate}
\section{Analisis exploratorio}
\subsection{Missing values}
\subsection{Ruidos y anomalias}
\subsection{Redundancia}
\subsection{Correlacion}
\section{Conclusiones}

\printbibliography

\end{document}