\documentclass[12pt, letterpaper, spanish]{article}

\usepackage{hyperref}
\usepackage{titling}
\usepackage[spanish]{babel} % Para palabras en espanhol
\usepackage{graphicx} %LaTeX package to import graphics
\graphicspath{{images/}} %configuring the graphicx package

% Preambulos
\title{\textbf{Universidad Catolica "Nuestra señora de la Asuncion" \\ [1ex] \large
Facultad de ciencias y tecnologia \\  Departamento de electronica e informatica}}

\author{Alain Vega}
\date{11 de septiembre del 2023}
% Logo de la catolica
\pretitle{%
  \begin{center}
  \LARGE
  \includegraphics[scale=0.6]{UcaLogo}\\[\bigskipamount]
}

\begin{document}
% Portada
\maketitle

\newpage
\tableofcontents % indice
\newpage

% Contenidos
\section{Caso de estudio: }
\subsection{Problema}
\section{Determinar objetivos del negocio}
\subsection{Background}
\subsection{Objetivos del negocio}
\subsection{Criterios de exito del negocio}
\section{Valoracion de la situacion}
\subsection{Inventario de recursos}
\subsection{Requisitos, supuestos y restricciones}
\subsection{Riesgos y Contingencias}
\subsection{Costes y Beneficios}
\section{Determinar los objetivos del DM}
\subsection{Metas de Data Mining}
\subsection{Criterios de éxito de DM}
\section{Referencias}
\begin{itemize}
    \item \href{https://www.sngular.com/es/crisp-dm-fase-i-comprension-del-negocio/?authuser=0}{Metodologia CRISP-DM}
    \item \href{https://www.datos.gov.py/search/type/dataset}{Datos abiertos paraguay}
\end{itemize}

\end{document}