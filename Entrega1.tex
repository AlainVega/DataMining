\documentclass[12pt, letterpaper, spanish]{article}

\usepackage{hyperref}
\usepackage{titling}
\usepackage[spanish]{babel} % Para palabras en espanhol
\usepackage{graphicx} %LaTeX package to import graphics
\graphicspath{{images/}} %configuring the graphicx package

% Preambulos
\title{\textbf{Universidad Catolica "Nuestra señora de la Asuncion" \\ 
[1ex] \large Facultad de ciencias y tecnologia \\ 
Departamento de electronica e informatica \\
\textit{Trabajo practico Entrega 1 - Mineria de datos}}}
\author{Alain Vega}
\date{11 de septiembre del 2023}
% Logo de la catolica
\pretitle{%
  \begin{center}
  \LARGE
  \includegraphics[scale=0.6]{UcaLogo}\\[\bigskipamount]
}

\begin{document}
% Portada
\maketitle

\newpage
\tableofcontents % indice
\newpage

% Contenidos
\section{Caso de estudio: Insuficiencia cardiaca}
\subsection{Problema}
Las enfermedades cardiovasculares (ECV) son la principal causa de muerte a nivel mundial, 
cobrando la vida de aproximadamente 17.9 millones de personas cada año, 
lo que representa el 31\% de todas las muertes en todo el mundo. 
Cuatro de cada 5 muertes por ECV son causadas por ataques cardíacos y accidentes cerebrovasculares, 
y un tercio de estas muertes ocurren prematuramente en personas menores de 70 años. 

La insuficiencia cardíaca es un evento común causado por las ECV.
Las personas con enfermedades cardiovasculares o que tienen un alto riesgo cardiovascular 
(debido a la presencia de uno o más factores de riesgo como hipertensión, diabetes, 
hiperlipidemia o enfermedades ya establecidas) necesitan una detección temprana y gestión, 
en la cual un modelo de aprendizaje automático puede ser de gran ayuda.
\section{Determinar objetivos del negocio}
\subsection{Background}

\subsection{Objetivos del negocio}
  \begin{itemize}
    \item{\textbf{Reducción de costos de atención médica: }
    al prevenir diagnósticos fallidos y tratamientos innecesarios, se reducen los costos
    asociados con el tratamiento de insuficiencia cardíaca y enfermedades relacionadas.}
    \item{\textbf{Mejorar la atención médica:} 
    al identificar de manera más efectiva a aquellos en riesgo de insuficiencia cardíaca. 
    Esto conlleva a un tratamiento más oportuno y eficaz.}
    \item{\textbf{Aumentar la calidad de vida de los pacientes:} 
    al permitirles recibir atención preventiva y seguimiento temprano gracias a la prediccion.}
  \end{itemize}
\subsection{Criterios de exito del negocio}
  \begin{itemize}
    \item{\textbf{Precision del modelo: }obtener un modelo con una fiabilidad del 90\%} 
    \item{\textbf{Reduccion de costos: }dismunir los costos medicos un x\%}
    \item{\textbf{Impacto en la vida de los pacientes: }afectar positivamente la calidad de vida
    de los pacientes.}
  \end{itemize}
\section{Valoracion de la situacion}
\subsection{Inventario de recursos}
Se cuenta con los siguientes datos medicos: \emph{edad, sexo, tipo de dolor en el pecho, 
presión arterial en reposo, colesterol, azúcar en sangre en ayunas, 
resultados del electrocardiograma en reposo, frecuencia cardíaca máxima alcanzada, 
angina inducida por el ejercicio, pico antiguo = ST, 
pendiente del segmento ST durante un ejercicio físico máximo en una prueba de esfuerzo cardíaco, 
sufrio de insuficiencia cardiaca.}

Que serviran para el entrenamiento y pruebas del modelo.
\subsection{Requisitos, supuestos y restricciones}
\begin{itemize}
  \item{\textbf{Requisitos: }deteccion temprana de posible insuficiencia cardiaca
  con una tasa de falsos negativos cada vez menor en proporcion a la
  cantidad de pacientes analizados.} 
  \item{\textbf{Supuestos: }los datos iniciales proveidos son de fiar, representan
  las caracteristicas principales a analizar.}
  \item{\textbf{Restricciones: }la fiabilidad del modelo nunca debe bajar del 90\%}
\end{itemize}
\subsection{Riesgos y Contingencias}
\begin{itemize}
  \item{\textbf{Riesgos: }} 
  \item{\textbf{Contingencias: }}
\end{itemize}
\subsection{Costes y Beneficios}
\section{Determinar los objetivos del DM}
\subsection{Metas de Data Mining}
\subsection{Criterios de éxito de DM}
\section{Bibliografia}
\begin{itemize}
    \item \href{https://www.sngular.com/es/crisp-dm-fase-i-comprension-del-negocio/?authuser=0}
    {Metodologia CRISP-DM}
    \item \href{https://www.datos.gov.py/search/type/dataset}
    {Conjunto de datos de insuficiencia cardiaca}
\end{itemize}

\end{document}